\documentclass{article}
\usepackage{amsmath, graphicx, amsfonts, caption, subfig, keyval, algorithm, algorithmic}
\usepackage{xcolor}
\captionsetup{justification=centering}
\newcommand{\R}{\mathbb{R}}
\newcommand{\C}{\mathcal{C}}
\newcommand{\X}{\mathcal{X}}
\DeclareMathOperator*{\argmin}{\arg\!\min}

\begin{document}
\title{AM221 Final Project Proposal}
\author{Taylor Killian \& Leonhard Spiegelberg}
\maketitle

\begin{abstract}
To be filled with an interesting summary of our work and results

\end{abstract}

%% Leos thoughts

\section{Dictionary Learning as 2-stage supermodular minimzation}
The introduced problem of Dictionary learning, which in its general form can be seen as
\[
\min_{D, R, \theta, \lambda} f(D, R, X) + g(D, \theta) + h(R, \lambda)
\]
where $f$ describes the objective function used to measure goodness of approximation of $X$ through $D, R$, $g$ describing suitable constraints on the dictionary, $h$ on the representation respectively.
With an input dataset $X=[x_1, \dots, x_k],  x_i \in \R^d,  X \in \R^{d\times k}$ we wish to find a dictionary $D \in \R^{d\times n},  \mathcal{D} = [d_1, \dots, d_n]$ and a representation $\mathcal{R}=[r_1,\dots,r_k], \ r_i\in\R^n, \ \mathcal{R}\in\R^{n\times k}$, such that both $\|X-\mathcal{D}\mathcal{R}\|_F^2$ is minimized and the representations $r_i$ are "sparse enough". To limit the dictionary becoming infinitely large (or small) we introduce a constraint on the dictionary's columns. For this any sufficient norm can be used. An often used norm is the $l_2$-norm. 
Thus for the dictionary learning problem we introduce the problem with constraints
\begin{alignat}{5}
         &\min_{D, R} \|X \ -D R\|_F^2  & \quad   \\
         &\text{s.t.}  \quad  &\|d_j\|_2 \leq 1&, \forall j=1, ...,n  \quad \\
         &  \quad  &\|r_i\|_0 \leq t&, \forall i=1, ...,k  \quad 
\end{alignat}

\textcolor{red}{Description of constraints here}


Using lagrange multipliers this can be brought to the general form above
\[
\min_{D, R, \theta, \lambda} \|X \ -D R\|_F^2 + \sum_{j=1}^m \theta_j (\| d_j\|_2 - 1)+ \sum_{i=1}^k \lambda_i (\| r_i \|_0 - t)
\]
I.e. the functions are
\[
\begin{split}
f(D, R, X) &:= \|X \ -D R\|_F^2 \\
g(D, \theta) &:= \sum_{j=1}^m \theta_j (\| d_j\|_2 - 1) \\
h(R, \lambda) &:= \sum_{i=1}^k \lambda_i (\| r_i \|_0 - t)
\end{split}
\]
\textcolor{red}{How to bring this to the penalty function form?}

\section{Introduction} \label{introduction}
Introduce the problem and it's complexities

\section{Background Information} \label{background}
\subsection{Background}
Here we'll want to add some information about Sparse regression, set functions and definitions of sub modular and super modular, Dictionary Selection, Greedy Methods, etc.

\subsection{Related Work}
A simple overview of prior work


\section{Methods} \label{methods}

Place holder section to begin to outline our work

\begin{algorithm}
  \caption{Notional algorithm}
  \label{alg:compass}
\begin{algorithmic}
  \STATE  Figure out optimal arrangement of rows and columns of input data
  \WHILE{there's still time in the semester}
  \IF{$f(p_k + s_k\lambda_i) < f(p_k)$ for some $\lambda_i$}
  \STATE $p_{k+1} = p_k + s_k\lambda_i$
  \STATE $s_{k+1} = s_k$
  \ELSE
  \STATE $p_{k+1} = p_{k}$
  \STATE $s_{k+1} = \alpha s_k$
  \ENDIF
  \ENDWHILE
  \RETURN VICTORIOUS
\end{algorithmic}
\end{algorithm}

\section{Experiments} \label{experiments}

Hold for applying our methods

\subsection{Results} \label{results}

The results of our algorithms against others

\section{Conclusions \& Future Work}
\subsection{Conclusions}
We will have done it!
 
\subsection{Future Work}
Rule the world 

\begin{thebibliography}{99}
  \bibitem{submod_spectral}
  Das, A., Kempe, D., (2011). Submodular meets Spectral: Greedy Algorithms for Subset Selection, Sparse Approximation and Dictionary Selection. \textit{Proceedings of the 28th International Conference on Machine Learnings}.
  \bibitem{greedy_selection}
  Cevher, V., Krause, A., (2011). Greedy Dictionary Selection for Sparse Representation. \textit{Selected Topics in Signal Processing, IEEE Journal of 5} (5), pp. 979 - 988.
  \bibitem{rIBP}
  Doshi-Velez, F., Williamson, S., (2015). Restricted Indian Buffet Processes. \textit{In submission}. arXiv:1508.06303 [stat.ME]
  
\end{thebibliography}
\end{document}